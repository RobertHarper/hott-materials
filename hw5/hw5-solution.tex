\documentclass[12pt]{article}

%%%%%%%%%%%%%%%%%%%%%%%%%%%%%%%%%%%%%%%%
% Basic packages
%%%%%%%%%%%%%%%%%%%%%%%%%%%%%%%%%%%%%%%%
\usepackage{amsmath,amsthm,amssymb}
\usepackage{mathtools}
\usepackage{etoolbox}
\usepackage{mathpartir}
\usepackage{verse}
\usepackage{stmaryrd}
\usepackage{hyperref}

%%%%%%%%%%%%%%%%%%%%%%%%%%%%%%%%%%%%%%%%
% Theorem-like environments
%%%%%%%%%%%%%%%%%%%%%%%%%%%%%%%%%%%%%%%%
\theoremstyle{plain}% default
\newtheorem{theorem}{Theorem}
\newtheorem{lemma}{Lemma}
\newtheorem{task}{Task}
\newtheorem{bonus}{Bonus Task}
\newenvironment{hint}{\noindent{\bf (Hint)}}{}

\theoremstyle{definition}
\newtheorem{definition}{Definition}

\theoremstyle{remark}
\newtheorem*{remark}{Remark}
%%%%%%%%%%%%%%%%%%%%%%%%%%%%%%%%%%%%%%%%
% HoTT
%%%%%%%%%%%%%%%%%%%%%%%%%%%%%%%%%%%%%%%%
\newrobustcmd*{\ctx}{\Gamma}
\newrobustcmd*{\entails}{\vdash}
\newrobustcmd*{\comp}{\mathbin{\circ}}
\newrobustcmd*{\jdeq}{\equiv} %%% judgmental equality
\newrobustcmd*{\defeq}{\vcentcolon\equiv} %%% definition. Sorry I screwed this up.
\newrobustcmd*{\id}{\mathtt{id}}
\newrobustcmd*{\dummy}{\underline{\hspace{0.5em}}}
\newrobustcmd*{\unknown}{\text{--}}
\mathchardef\mhyphen="2D
\newrobustcmd*{\blank}{\underline{\hspace{2em}}}
% Units
\newrobustcmd*{\triv}{\langle\rangle}
%\newrobustcmd*{\triv}{\star} -- the star version
% Bottoms
\newrobustcmd*{\abort}{\mathtt{abort}}
% Products
\newrobustcmd*{\fst}{\mathtt{fst}}
\newrobustcmd*{\snd}{\mathtt{snd}}
\newrobustcmd*{\bracket}[1]{\langle #1\rangle} % alternative
\newrobustcmd*{\pair}[2]{\bracket{#1,#2}}
\newrobustcmd*{\tuple}[2]{\bracket{#1,#2}} % alternative
\newrobustcmd*{\tuplepair}[2]{\bracket{#1,#2}} % alternative
\newrobustcmd*{\triple}[3]{\bracket{#1,#2,#3}}
\newrobustcmd*{\productlevel}{\mathtt{product\mhyphen level}}
% Sums
\newrobustcmd*{\inl}{\mathtt{inl}}
\newrobustcmd*{\inr}{\mathtt{inr}}
\newrobustcmd*{\case}{\mathtt{case}}
\newrobustcmd*{\sigmalevel}{\mathtt{sigma\mhyphen level}}
% Natural numbers
\newrobustcmd*{\rec}{\mathtt{rec}}
\newrobustcmd*{\zero}{\mathtt{zero}}
\newrobustcmd*{\suc}{\mathtt{suc}}
\newrobustcmd*{\N}{\mathbb{N}}
\newrobustcmd*{\nat}{\mathtt{nat}}
% Identities
\newrobustcmd*{\Id}{\mathtt{Id}}
\newrobustcmd*{\refl}{\mathtt{refl}}
\newrobustcmd*{\J}{\mathsf{J}}
\newrobustcmd*{\sym}{\mathtt{sym}}
\newrobustcmd*{\trans}{\mathtt{trans}}
\newrobustcmd*{\ap}{\mathtt{ap}}
\newrobustcmd*{\tr}{\mathtt{tr}}
%%% Path concatenation
\newrobustcmd*{\concat}{%
  \mathchoice{\mathbin{\raisebox{0.5ex}{$\displaystyle\centerdot$}}}%
  {\mathbin{\raisebox{0.5ex}{$\centerdot$}}}%
  {\mathbin{\raisebox{0.25ex}{$\scriptstyle\,\centerdot\,$}}}%
  {\mathbin{\raisebox{0.1ex}{$\scriptscriptstyle\,\centerdot\,$}}}
}
% Universes
\newrobustcmd*{\U}{\mathcal{U}}
\newrobustcmd*{\ua}{\mathtt{ua}}
\newrobustcmd*{\uaequiv}{\mathtt{ua\mhyphen equiv}}
% Functional Extensionality
\newrobustcmd*{\funext}{\mathtt{funext}}
\newrobustcmd*{\happly}{\mathtt{happly}}
% Equivalences
\newrobustcmd*{\eqv}{\simeq} % from HoTT-book
\newrobustcmd*{\qinv}{\mathsf{qinv}}
\newrobustcmd*{\isequiv}{\mathsf{isEquiv}}
\newrobustcmd*{\ecomp}{\mathbin{\comp_\eqv}}
\newrobustcmd*{\equivinv}{\mathtt{equiv\mhyphen inv}}
\newrobustcmd*{\equivcompose}{\mathtt{equiv\mhyphen compose}}
\newrobustcmd*{\equiviscontr}{\mathtt{equiv\mhyphen is\mhyphen contr}}
\newrobustcmd*{\equivlevel}{\mathtt{equiv\mhyphen level}}
\newrobustcmd*{\equivpreserveslevel}{\mathtt{equiv\mhyphen preserves\mhyphen level}}
\newrobustcmd*{\isequivisprop}{\mathtt{is\mhyphen equiv\mhyphen is\mhyphen prop}}
% N-types
\newrobustcmd*{\tlevel}{\mathtt{tlevel}} % 't' for truncation
\newrobustcmd*{\trec}{\rec_\tlevel} % 't' for truncation
\newrobustcmd*{\trunc}[1]{\left\|#1\right\|}
\newrobustcmd*{\squash}{\mathtt{squash}}
\newrobustcmd*{\truncproj}[1]{\left|#1\right|}
\newrobustcmd*{\truncelim}{\mathtt{elim}_\mathtt{trunc}}
\newrobustcmd*{\isContr}{\mathtt{isContr}}
\newrobustcmd*{\iscontr}{\mathtt{isContr}}
\newrobustcmd*{\isProp}{\mathtt{isProp}}
\newrobustcmd*{\isprop}{\mathtt{isProp}}
\newrobustcmd*{\isSet}{\mathtt{isSet}}
\newrobustcmd*{\isset}{\mathtt{isSet}}
\newrobustcmd*{\istype}[1]{\mathtt{is}\mhyphen#1\mhyphen\mathtt{type}}
\newrobustcmd*{\propisset}{\mathtt{prop\mhyphen is\mhyphen set}}
\newrobustcmd*{\contrequiv}{\mathtt{contr\mhyphen equiv}}
\newrobustcmd*{\proplevel}{\mathtt{prop\mhyphen level}}
\newrobustcmd*{\contrisprop}{\mathtt{contr\mhyphen is\mhyphen prop}}
\newrobustcmd*{\raisecontr}{\mathtt{raise\mhyphen contr}}
\newrobustcmd*{\raiselevel}{\mathtt{raise\mhyphen level}}
\newrobustcmd*{\prophaslevelsuc}{\mathtt{prop\mhyphen has\mhyphen level\mhyphen\suc}}
\newrobustcmd*{\istypeisprop}{\mathtt{is\mhyphen type\mhyphen is\mhyphen prop}}
\newrobustcmd*{\ntypepathequiv}{\mathtt{ntype\mhyphen path\mhyphen equiv}}
% Subtypes
\newrobustcmd*{\subtypepath}{\mathtt{subtype\mhyphen path}}
\newrobustcmd*{\subtypepatheqiv}{\mathtt{subtype\mhyphen path\mhyphen equiv}}
\newrobustcmd*{\subtypelevel}{\mathtt{subtype\mhyphen level}}

%%%%%%%%%%%%%%%%%%%%%%%%%%%%%%%%%%%%%%%%
% Equational Reasoning
%%%%%%%%%%%%%%%%%%%%%%%%%%%%%%%%%%%%%%%%
\newrobustcmd*{\reason}[1]{=\hspace{-0.5em}\langle\hspace{0.2em} #1 \hspace{0.2em}\rangle}


\usepackage{palatino}
\usepackage{mdframed}
\usepackage[usenames,svgnames]{xcolor}
\newcommand{\cut}[1]{}
\def\StripPrefix#1>{}
\def\jobis#1{FF\fi
  \def\predicate{#1}%
  \edef\predicate{\expandafter\StripPrefix\meaning\predicate}%
  \edef\job{\jobname}%
  \ifx\job\predicate
}
\if\jobis{hw5-solution}%
  \newcommand{\marksolution}[1]{\color{FireBrick}#1\normalcolor}%
\else%
  \newcommand{\marksolution}[1]{\cut{#1}}%
\fi
\theoremstyle{plain}
\newtheorem*{solution}{Solution}
\newrobustcmd{\markextra}[1]{\color{DarkOliveGreen}#1\normalcolor}
\mdfdefinestyle{extra}{linecolor=DarkOliveGreen,fontcolor=DarkOliveGreen}

\title{\Large\textbf{
  Homework 5: Leap of Truncation}
\normalsize\\
15-819 Homotopy Type Theory\\
Favonia (\href{mailto:favonia@cmu.edu}{favonia@cmu.edu})}

\author{}

\date{%
Out: 19/Nov/13\\
Due: 4/Dec/13
}

\begin{document}

\maketitle

\section{Blah, Blah, Blah (a.k.a. Introduction)}

Alright. So, the goal of this assignment is to show that
the universe of $n$-types is itself an $(n{+}1)$-type.
Every task in this assignment is a part of the proof of this theorem.
Let us begin right away with the definition of \emph{universes of $n$-types}:
\begin{definition}
  The
  \emph{universe of $n$-types}
  is
  $\displaystyle \U^n \defeq \sum_{A:\U} \istype{n}(A)$.%
  \footnote{$\U_i$ is used for the universe level so I have to find another corner. Sorry.}
\end{definition}

To formally talk about the $n$, the \emph{truncation level},
let us have a separate data type, $\tlevel$,
which is isomorphic to $\nat$ but starts with $-2$.
We write $\trec$ as the recursor,
and reuse $\suc$ as the successor constructor.
For example, the $(n{+}1)$ is really $\suc(n)$ in the formal statement,
and $-1$ is really $\suc(-2)$.
Anyway, here is the theorem:
\begin{theorem}
  $\displaystyle \prod_{n:\tlevel} \istype{\suc(n)}(\U^n)$.
\end{theorem}

The key insight is that
the type of equivalences between two $n$-types
is itself an $n$-type.
By univalence,
it means that the type of paths between two $n$-types
is also an $n$-type.
The intuition behind the key insight is that
equivalences are maps with special properties,
and the structure of parallel maps are, to some degree,
limited by that of the codomoin in the presence of functional extensionality.
You will fill in the missing details in your solution.

As a side note,
this theorem is actually tight for $n \geq -1$, in the sense that
$\U^n$ is not a $n$-type for $n \geq -1$.%
\footnote{It has been known that $\U^{-1}$ is not a $(-1)$-type (i.e., not a proposition)
  and $\U^0$ is not a $0$-type (i.e. not a set) for a while,
  and Nicolai Kraus and Christian Sattler generalized this to $n\geq -1$
  without using higher inductive types.}
We will not cover this result in this assignment.
\markextra{
  \begin{mdframed}[style=extra]
    \textsc{Clarification}: The bound was $n \geq 0$ but it works for $n = -1$ too.
  \end{mdframed}
}

To make programming easier, you are allowed to use pattern matching
for $\Sigma$ types in $\lambda$.
The dedicated syntax $\lambda \pair{x}{y}.M$
means $\lambda p.[\fst(p),\snd(p)/x,y]M$
for some fresh variable $p \notin M$.

\section{Cumulativeness}

In this section
we will show that the truncation hierarchy is cumulative,
along with some other useful lemmas.
You may assume the following lemma that was proved in class.
\[
  \displaystyle \propisset : \prod_{A:\U} \isprop(A) \to \isset(A)
\]
As usual, you do not have to justify your code.

\begin{task}
  $\displaystyle
    \proplevel : \prod_{A : \U} \isprop(A) \to \istype{(-1)}(A)
  $
\end{task}
\marksolution{
\begin{solution}
  $
    \lambda A \alpha x y.\pair{\alpha(x)(y)}{\propisset(A)(\alpha)(x)(y)(\alpha(x)(y))}
  $
\end{solution}
}

\begin{task}
  $\displaystyle
    \contrisprop : \prod_{A : \U} \iscontr(A) \to \isprop(A).
  $
\end{task}
\marksolution{
\begin{solution}
  $
    \lambda A\tuple{a}{\alpha} x y.\alpha(x)^{-1} \concat \alpha(y)
  $
\end{solution}
}

\begin{task}
  $\displaystyle
    \raiselevel :
    \prod_{(A:\U)}
    \prod_{(n:\tlevel)}
    \istype{n}(A) \to \istype{\suc(n)}(A)
  $
\end{task}
\marksolution{
\begin{solution}
  \begin{multline*}
    \lambda An.\trec[n.\prod_{A:\U}\istype{n}(A) \to \istype{\suc(n)}(A)]\\
    (n,\lambda A.\proplevel(A) \comp \contrisprop(A),n.f.\lambda A\alpha xy.f(x=y)(\alpha(x)(y)))(A)
  \end{multline*}
  (Note that to be fully consistent with the lecture notes, $\trec$ should be $\mathtt{ind}_\mathtt{level}$.)
\end{solution}
}

\begin{task}
  $\displaystyle
  \prophaslevelsuc :
  \prod_{(A:\U)}
  \prod_{(n:\tlevel)}
  \isprop(A) \to \istype{\suc(n)}(A)
  $
\end{task}
\marksolution{
  \begin{solution}
    \begin{multline*}
      \lambda An\alpha.\trec[n.\istype{\suc(n)}(A)]\\
      (n,\proplevel(A)(\alpha),n.\beta.\raiselevel(A)(\suc(n))(\beta))
    \end{multline*}
    (Note that to be fully consistent with the lecture notes, $\trec$ should be $\mathtt{ind}_\mathtt{level}$.)
  \end{solution}
}

\section{Equivalences}

This section is to show the ``key insight'',
i.e., that the type of equivalences between $n$-types is an $n$-type.
\[
  \prod_{(A,B:\U)}
  \prod_{(n:\tlevel)}
  \istype{n}(A) \to \istype{n}(B) \to \istype{n}(A \eqv B)
\]

The strategy is to exploit the fact that $\isequiv(f)$ is a proposition.
In general we call $\sum_{x:A} B(x)$ a \emph{subtype} of $A$
when $\prod_{x:A} \isprop(B(x))$.
The idea is that the truncation level of a subtype
should be bounded by that of its original type.
This is true except one special case mentioned below.
In the case of equivalences, this means that
you can disregard the second component $\isequiv(f)$
and only focus on the function $f$.
Moreover, we know the truncation level of the function type is bounded by
that of the codomain, and hence complete the proof.
Unfortunately this strategy can fail when $A$ is contractible,
because a subtype of a contractible type might not be contractible.
To address, we will handle the case $n = -2$ separately.
You may assume these lemmas:
\begin{itemize}
  \item Some facts derived from the function extensionality.
    Suppose $f, g: \prod_{x:A} B(x)$ for some $A : \U$ and $B : A \to \U$.
    \[
      \funext : \left(\prod_{x:A} f\,x = g\,x \right) \to (f = g)
    \]
    You do not have to write out implicit arguments $A$, $B$, $f$ and $g$.
  \item Some operations of $\isequiv(f)$ so that its implementation becomes irrelevant.
    Suppose $A, B, C : \U$.
    \begin{gather*}
      \equivinv : A \eqv B \to B \eqv A
      \\
      \equivcompose : (B \eqv C) \to (A \eqv B) \to (A \eqv C)
      \\
      \isequivisprop : \prod_{f:A\to B}\isprop(\isequiv(f))
    \end{gather*}
    Note that $\equivcompose$ is in the function composition order,
    which can also be written as $e_1 \ecomp e_2$,
    mimicking the function composition $f \comp g$.
    You do not have to write out implicit arguments $A$, $B$ and $C$.
  \item Truncation levels are closed under $\prod$ and $\sum$ formations.
    Suppose $A : \U$, $B : A \to \U$ and $n : \tlevel$.
    \[
      \productlevel :
      \left(\prod_{x:A}\istype{n}(B(x))\right) \to \istype{n}\left(\prod_{x:A}B(x)\right)
    \]
    \begin{multline*}
      \sigmalevel :
      \istype{n}(A)
      \to
      \left(\prod_{x:A}\istype{n}(B(x))\right)
      \\
      \to
      \istype{n}\left(\sum_{x:A}B(x)\right)
    \end{multline*}
    You do not have to write out implicit arguments $A$, $B$ and $n$.
  \item Finally, you may disregard the second component while handling paths in subtypes.
    Suppose $A : \U$ and $B : A \to \U$.
    \begin{gather*}
      \subtypepath :
      \left(\prod_{x{:}A} \isprop(B(x))\right)
      \to
      \prod_{m,n:\sum_{x:A}B(x)}
      (\fst(m)=\fst(n))
      \to
      (m = n)
      \\
      \subtypepatheqiv :
      \left(\prod_{x{:}A} \isprop(B(x))\right)
      \to
      \prod_{m,n:\sum_{x:A}B(x)}
      (\fst(m)=\fst(n))
      \eqv
      (m = n)
    \end{gather*}
    Again, you do not have to write out implicit arguments $A$ and $B$.
\end{itemize}

\subsection{The Special Case}

Let us begin with the special case $n = -2$.
\begin{task}
  $\displaystyle
  \contrequiv :
  \prod_{A,B:\U}
  \iscontr(A) \to
  \iscontr(B) \to
  (A \eqv B)
  $
\end{task}
\markextra{
  \begin{mdframed}[style=extra]
    \textsc{Erratum}: There was a typo that $B : A \to \U$. It should be $B : \U$.
  \end{mdframed}
}
\marksolution{
  \begin{solution}
    $
    \lambda AB\pair{a}{\alpha} \pair{b}{\beta}.\triple{\lambda\dummy.b}{
      \pair{\lambda\dummy.a}{\beta}
    }{
      \pair{\lambda\dummy.a}{\alpha}
    }
    $
  \end{solution}
}
\begin{task} Implement
  \[
    \equiviscontr :
    \prod_{A,B:\U}
    \iscontr(A) \to \iscontr(B) \to \iscontr(A \eqv B)
  \]
  Warning: this can be mind-blowing.
  \begin{hint}
    Bob and Favonia have office hours.
  \end{hint}
\end{task}
\marksolution{
  \begin{solution}
    \begin{multline*}
      \lambda AB\alpha\beta.\pair{
        \contrequiv(A)(B)(\alpha)(\beta)
      }{
        \lambda \gamma.\subtypepath(\isequivisprop)
        \\
        (\contrequiv(A)(B)(\alpha)(\beta))
      (\gamma)(\funext(\lambda a.\snd(\beta)(\fst(\gamma)(a)))}
    \end{multline*}
  \end{solution}
}

\subsection{Other Cases}

Now let us consider the cases $n \geq -1$.
The strategy outlined in the beginning of the section critically depends on the following lemma.
\begin{task} Implement
  \begin{multline*}
    \subtypelevel :
    \prod_{(A:\U)}
    \prod_{(B:A\to\U)}
    \prod_{(n:\tlevel)}
    \istype{\suc(n)}(A)
    \to
    \left(\prod_{x:A} \isprop(B(x))\right)
    \\
    \to
    \istype{\suc(n)}\left(\sum_{x:A}B(x)\right)
  \end{multline*}
\end{task}
\marksolution{
  \begin{solution}
    \begin{multline*}
      \lambda A B n \alpha \beta.\sigmalevel(\alpha)(\lambda x.\prophaslevelsuc(B(x))(n)(\beta(x)))
    \end{multline*}
  \end{solution}
}

\begin{bonus}
  Find a counterexample of the above task if $\istype{\suc(n)}$ was $\istype{n}$.
  That is, find $A$ and $B$ such that $\iscontr(A)$ but $\neg \iscontr(\sum_{x:A}B(x))$.
\end{bonus}
\marksolution{
  \begin{solution}
    Let $A = \top$ and $B = \lambda\dummy.\bot$.
    We have
    \[
      \tuple{\triv}{\lambda\dummy.\refl(\triv)}:\iscontr(A)
    \]
    and
    \[
      \textstyle
      \snd \comp \fst : \iscontr\left(\sum_{x:A}B(x)\right) \to \bot
    \]
  \end{solution}
}

Finally the theorem of this section.
\begin{task} Implement
  \[
    \equivlevel :
    \prod_{(A,B:\U)}
    \prod_{(n:\tlevel)}
    \istype{n}(A) \to \istype{n}(B) \to \istype{n}(A \eqv B)
  \]
\end{task}
\marksolution{
  \begin{solution}
    \begin{multline*}
      \lambda ABn.\trec[
        n.\istype{n}(A) \to \istype{n}(B) \to \istype{n}(A \eqv B)
      ]
      \\
      (n,\equiviscontr(A)(B),n.\dummy.\lambda\dummy\beta.
      \subtypelevel(A\to B)(\isequiv)(n)
      \\
      (\productlevel(\lambda\dummy.\beta))(\isequivisprop))
    \end{multline*}
    (Note that to be fully consistent with the lecture notes, $\trec$ should be $\mathtt{ind}_\mathtt{level}$.)
  \end{solution}
}

\section{Universes}

We are reaching the ultimate goal.
Feel free to use any lemma or task in previous sections, in addition to these:
\begin{itemize}
  \item $\istype{n}(A)$ is also a proposition.
    \[
      \istypeisprop :
      \prod_{(n:\tlevel)}
      \prod_{(A:\U)}
      \isprop(\istype{n}(A))
    \]
  \item
    Some facts derived from the univalence axiom.
    Suppose $A, B : \U$.
    \begin{gather*}
      \ua : A \eqv B \to A =_\U B
      \\
      \uaequiv : (A \eqv B) \eqv (A =_\U B)
    \end{gather*}
    As usual, you do not have to write out implicit arguments $A$ and $B$.
\end{itemize}

\subsection{Real Work}

\begin{task}
  Everything respects equivalences, thanks to the univalence axiom.
  \[
    \equivpreserveslevel :
    \prod_{(A,B:\U)}
    \prod_{(n:\tlevel)}
    (A \eqv B) \to \istype{n}(A) \to \istype{n}(B)
  \]
\end{task}
\marksolution{
  \begin{solution}
    $
    \lambda A B n e\alpha . \tr[x.\istype{n}(x)](\ua(e))(\alpha)
    $
  \end{solution}
}


\begin{task} Implement
  \[
    \ntypepathequiv :
    \prod_{(n:\tlevel)}
    \prod_{(A,B:\U^n)}
    (\fst(A) =_\U \fst (B))
    \eqv
    (A =_{\U^n} B)
  \]
  \begin{hint}
    The lemma name is suggestive. Maybe too suggestive.
  \end{hint}
\end{task}
\marksolution{
  \begin{solution}
    $
    \lambda n. \subtypepatheqiv(\istypeisprop(n))
    $
  \end{solution}
}

\begin{task}
  Show the ultimate theorem and finish this assignment.
\end{task}
\marksolution{
  \begin{solution}
    \begin{multline*}
      \lambda n\pair{A}{\alpha}\pair{B}{\beta}.
      \equivpreserveslevel(A\eqv B)(\pair{A}{\alpha}=\pair{B}{\beta})(n)
      \\
      (\ntypepathequiv(n)(\pair{A}{\alpha})(\pair{B}{\beta}) \ecomp \uaequiv)
      \\
      (\equivlevel(A)(B)(n)(\alpha)(\beta))
    \end{multline*}
  \end{solution}
}
\end{document}
