\documentclass[12pt]{article}

%%%%%%%%%%%%%%%%%%%%%%%%%%%%%%%%%%%%%%%%
% Basic packages
%%%%%%%%%%%%%%%%%%%%%%%%%%%%%%%%%%%%%%%%
\usepackage{amsmath,amsthm,amssymb}
\usepackage{mathtools}
\usepackage{etoolbox}
\usepackage{mathpartir}
\usepackage{verse}
\usepackage{hyperref}

%%%%%%%%%%%%%%%%%%%%%%%%%%%%%%%%%%%%%%%%
% Theorem-like environments
%%%%%%%%%%%%%%%%%%%%%%%%%%%%%%%%%%%%%%%%
\theoremstyle{plain}% default
\newtheorem{theorem}{Theorem}[section]
\newtheorem{lemma}{Lemma}[section]
\newtheorem{task}{Task}
\newtheorem{bonus}{Bonus Task}
\newenvironment{hint}{\noindent{\bf (Hint)}}{}

\theoremstyle{definition}
\newtheorem{definition}{Definition}[section]

\theoremstyle{remark}
\newtheorem*{remark}{Remark}
%%%%%%%%%%%%%%%%%%%%%%%%%%%%%%%%%%%%%%%%
% ITT + HoTT
%%%%%%%%%%%%%%%%%%%%%%%%%%%%%%%%%%%%%%%%
\newrobustcmd*{\ctx}{\Gamma}
\newrobustcmd*{\entails}{\vdash}
\newrobustcmd*{\comp}{\mathbin{\circ}}
\newrobustcmd*{\jdeq}{\equiv} %%% judgmental equality
\newrobustcmd*{\defeq}{\vcentcolon\equiv} %%% definition. Sorry I screwed this up.
\newrobustcmd*{\id}{\mathtt{id}}
\newrobustcmd*{\dummy}{\underline{\hspace{0.5em}}}
\newrobustcmd*{\unknown}{\text{--}}
\newrobustcmd*{\blank}{\underline{\hspace{2em}}}
% Units
\newrobustcmd*{\triv}{\langle\rangle}
%\newrobustcmd*{\triv}{\star} -- the star version
% Bottoms
\newrobustcmd*{\abort}{\mathtt{abort}}
% Products
\newrobustcmd*{\fst}{\mathtt{fst}}
\newrobustcmd*{\snd}{\mathtt{snd}}
\newrobustcmd*{\pair}[2]{\langle #1,#2\rangle}
\newrobustcmd*{\tuple}[2]{\langle #1,#2\rangle} % alternative
\newrobustcmd*{\tuplepair}[2]{\langle #1,#2\rangle} % alternative
% Sums
\newrobustcmd*{\inl}{\mathtt{inl}}
\newrobustcmd*{\inr}{\mathtt{inr}}
\newrobustcmd*{\case}{\mathtt{case}}
% Natural numbers
\newrobustcmd*{\rec}{\mathtt{rec}}
\newrobustcmd*{\zero}{\mathtt{zero}}
\newrobustcmd*{\suc}{\mathtt{suc}}
\newrobustcmd*{\N}{\mathbb{N}}
\newrobustcmd*{\nat}{\mathtt{nat}}
% Identities
\newrobustcmd*{\Id}{\mathtt{Id}}
\newrobustcmd*{\refl}{\mathtt{refl}}
\newrobustcmd*{\J}{\mathsf{J}}
\newrobustcmd*{\sym}{\mathtt{sym}}
\newrobustcmd*{\trans}{\mathtt{trans}}
\newrobustcmd*{\ap}{\mathtt{ap}}
\newrobustcmd*{\tr}{\mathtt{tr}}
%%% Path concatenation
\newrobustcmd*{\concat}{%
  \mathchoice{\mathbin{\raisebox{0.5ex}{$\displaystyle\centerdot$}}}%
  {\mathbin{\raisebox{0.5ex}{$\centerdot$}}}%
  {\mathbin{\raisebox{0.25ex}{$\scriptstyle\,\centerdot\,$}}}%
  {\mathbin{\raisebox{0.1ex}{$\scriptscriptstyle\,\centerdot\,$}}}
}
% Universes
\newrobustcmd*{\U}{\mathcal{U}}
% Equivalences
\newrobustcmd*{\eqv}{\simeq} % from HoTT-book
\newrobustcmd*{\qinv}{\mathsf{qinv}}
%%%%%%%%%%%%%%%%%%%%%%%%%%%%%%%%%%%%%%%%
% Names of lemmas
%%%%%%%%%%%%%%%%%%%%%%%%%%%%%%%%%%%%%%%%
\mathchardef\mhyphen="2D
\newrobustcmd*{\apconcat}{\mathtt{ap\mhyphen concat}}
\newrobustcmd*{\apid}{\mathtt{ap\mhyphen id}}
\newrobustcmd*{\apcomp}{\mathtt{ap\mhyphen comp}}
\newrobustcmd*{\lemmaone}{\mathtt{lemma}_1}
\newrobustcmd*{\lemmatwo}{\mathtt{lemma}_2}
%%%%%%%%%%%%%%%%%%%%%%%%%%%%%%%%%%%%%%%%
% Equational reasoning
%%%%%%%%%%%%%%%%%%%%%%%%%%%%%%%%%%%%%%%%
\newrobustcmd*{\reason}[1]{=\hspace{-0.5em}\langle\hspace{0.2em} #1 \hspace{0.2em}\rangle}
\newrobustcmd*{\concatassoc}{\mathtt{concat\mhyphen assoc}}


\usepackage{palatino}
\usepackage{mdframed}
\usepackage[usenames,svgnames]{xcolor}
\newcommand{\cut}[1]{}
\def\StripPrefix#1>{}
\def\jobis#1{FF\fi
  \def\predicate{#1}%
  \edef\predicate{\expandafter\StripPrefix\meaning\predicate}%
  \edef\job{\jobname}%
  \ifx\job\predicate
}
\if\jobis{hw4-solution}%
  \newcommand{\marksolution}[1]{\color{FireBrick}#1\normalcolor}%
\else%
  \newcommand{\marksolution}[1]{\cut{#1}}%
\fi
\newenvironment{solution}{\trivlist \item[\hskip \labelsep{\bf
Solution:}]}{\endtrivlist}
\newrobustcmd{\markextra}[1]{\color{DarkOliveGreen}#1\normalcolor}
\mdfdefinestyle{extra}{linecolor=DarkOliveGreen,fontcolor=DarkOliveGreen}

\title{\Large\textbf{
  Homework 4: Second Identity Crisis}
\normalsize\\
15-819 Homotopy Type Theory}

\author{}

\date{%
Out: 4/Nov/13\\
Due: 18/Nov/13
}

\begin{document}

\maketitle

\section{Application Programming Interface}

Assume
$\ctx \entails f : A \to B$ and
$\ctx \entails g : B \to C$.
Here are some useful properties of $\ap$.
\begin{align*}
  &\ctx \entails \blank : \prod_{m:A}\ap_f(\refl_A(m)) = \refl_B(f\,m)
  \\
  &\ctx \entails \blank : \prod_{m,n:A}\prod_{p:m=n}\ap_f(p^{-1}) = \ap_f(p)^{-1}
  \\
  &\ctx \entails \apconcat : \prod_{l,m,n:A}\prod_{p:l=m}\prod_{q:m=n}\ap_f(p \concat q) = \ap_f(p) \concat \ap_f(q)
  \\
  &\ctx \entails \blank : \prod_{m,n:A}\prod_{p:m=n}\ap_{\id_A}(p) = p
  \\
  &\ctx \entails \apcomp : \prod_{m,n:A}\prod_{p:m=n}\ap_{g \comp f}(p) = \ap_g(\ap_f(p))
\end{align*}
% Basically these say that $\ap_f(p)$ is a bifunctor with respect to $f$ and $p$.

\pagebreak[2]

\begin{task}
  Implement $\apconcat$ and $\apcomp$.
  \begin{align*}
    &\ctx \entails \apconcat : \prod_{l,m,n:A}\prod_{p:l=m}\prod_{q:m=n}\ap_f(p \concat q) = \ap_f(p) \concat \ap_f(q)
    \\
    &\ctx \entails \apcomp : \prod_{m,n:A}\prod_{p:m=n}\ap_{g \comp f}(p) = \ap_g(\ap_f(p))
  \end{align*}
  You may assume that $\refl(m) \concat p \jdeq p$, $\refl(m)^{-1} \jdeq \refl(m)$,
  and $\ap_f(\refl(m)) \jdeq \refl(f(m))$.
  \begin{hint}
    Make sure that your motives are well-typed
    as you did for the associativity in Homework~3.
  \end{hint}
\end{task}

\marksolution{
\begin{solution}
  All the functions:
  \begin{align*}
    &\,
    \lambda m.\refl_{f(m)=f(m)}(\refl_B(f(m)))
    \\
    &\,
    \lambda mnp. \J[m.n.p.\ap_f(p^{-1}) = \ap_f(p)^{-1}](p, x.\refl_{f\,m=f\,m}(\refl_B(f(m))))
    \\
    \apconcat
    \defeq
    &\,
    \textstyle
    \lambda lmnp. \J[l.m.p.\prod_q\ap_f(p \concat q) = \ap_f(p) \concat \ap_f(q)](p,
    m.\lambda q.\refl_{f(m)=f(n)}(\ap_f(q)))
    \\
    &\,
    \lambda mnp.\J[m.n.p.\ap_{\id_A}(p) = p](p, m.\refl_{m=m}(\refl_A(m)))
    \\
    \apcomp
    \defeq
    &\,
    \lambda mnp.\J[m.n.p.\ap_{g \comp f}(p) = \ap_g(\ap_f(p))](p, m.\refl_{g(f(m))=g(f(m))}(\refl_C(g(f(m))))
  \end{align*}
\end{solution}
}

\section{Paths to Infinity}
\begin{verse}
  \textit{It's one thing to feel that you are on the right path, but it's another to think yours is the only path.}%
  \footnote{This quote has been contributed to Paulo Coelho on the Internet, but I could not locate the source.}
\end{verse}

One of the common tasks in HoTT is to
understand the structure of the path space $\Id_A(m,n)$
of a particular space $A$,
with the goal to find a seemingly more manageable type equivalent to $\Id_A(m,n)$,
which is to show that
\[
  \prod_{x,y:A} \Id_A(x,y) \eqv F(x,y)
\]
for some ``explicit'' $F$.
Recall that the data format of an equivalence is
$(f,(g,\alpha),(h,\beta))$
where $\alpha$ is a witness of $g$ being a right-inverse of $f$
and $\beta$ is of $h$ being a left-inverse.

\subsection{Tops and Bottoms are Dual in Many Ways}

\begin{task}
  Show that $\prod_{x,y:\top} \Id_\top(x,y) \eqv \top$
  (where $F(x,y) \defeq \top$).
  You do not have to justify your code.
  \begin{hint}
    What is the $\eta$ rule of $\top$?
  \end{hint}
\end{task}

\marksolution{
  \begin{solution}
    $\lambda xy.(f,(g,\alpha),(h,\beta))$ with the following data:
    \[
      \begin{array}{lcl}
        f &\defeq& \lambda \dummy . \triv
        \\
        g &\defeq& \lambda \dummy . \refl_\top(\triv)
        \\
        \alpha &\defeq& \lambda \dummy . \refl_\top(\triv)
        \\
        h &\defeq& \lambda \dummy . \refl_\top(\triv)
        \\
        \beta &\defeq& \lambda p . \J[\dummy.\dummy.p.\refl_\top(\triv)=p](p,\dummy.\refl_{\triv=\triv}(\refl_\top(\triv)))
      \end{array}
    \]
  \end{solution}
}

\begin{task}
  Show that $\prod_{x,y:\bot} \Id_\bot(x,y) \eqv \bot$
  (where $F(x,y) \defeq \bot$).
  You do not have to justify your code.
  \begin{hint}
    Favonia can answer this with less than 30 characters.
  \end{hint}
\end{task}

\marksolution{
  \begin{solution}
    $\displaystyle
      \lambda xy.\abort[\Id_\bot(x,y) \eqv \bot](x)
    $
  \end{solution}
}

\subsection{Pathbreaking Techniques}

Now let's focus on the product type.
In class we almost finished the characterization
except the definition of $\ap^2$ and some critical lemmas.
As a reminder, the goal is to show that
\[
  \prod_{x,y:A\times B} \Id_{A\times B}(x,y) \eqv \Id_A(\fst(x),\fst(y)) \times \Id_B(\snd(x),\snd(y)).
\]

There are many, many ways to define $\ap^2$
but we will only cover two special ones.
Their particularity is that the Eckmann-Hilton theorem
can be trivially derived from the equivalence between them!
One of them is shown below,
assuming that $f : C \to D \to E$,
$p : c_1 =_C c_2$, and $q : d_1 =_D d_2$
in the current context.
\[
  \ap^2_f(p,q) \defeq \ap_{\lambda c.f(c)(d_1)}(p) \concat \ap_{f(c_2)}(q)
\]
\begin{task}
  \label{task:ap2prime}
  Find another one which is ``symmetric'' to the above implementation,
  where $q$ comes before $p$ instead of $p$ coming before $q$.
  (We will refer to this implementation as $\ap'^2_f$.)
\end{task}
\marksolution{
  \begin{solution}
    $\displaystyle
    \ap'^2_f(p,q) \defeq \ap_{f(c_1)}(q) \concat \ap_{\lambda c.f(c)(d_2)}(p)
    $
  \end{solution}
}

In order to complete the characterization, for every $x,y : A \times B$
we need to prepare the data $(f,(g,\alpha),(h,\beta))$ for the equivalence.
We already implemented $f$, $g$ and $h$ as follows:
\[
  \begin{array}{lcl}
    f &\defeq& \lambda p.\pair{\ap_\fst(p)}{\ap_\snd(p)}
    \\
    g &\defeq& \lambda q.\ap^2_{\lambda ab.\pair{a}{b}}(\fst(q),\snd(q))
    \\
    h &\defeq& g
  \end{array}
\]
The remaining are $\alpha$ and $\beta$. The $\beta$ of type
\[
  \prod_{p : \Id_{A\times B}(x,y)} h(f(p)) = p
\]
is relatively easy because the domain is $\Id_{A\times B}(x,y)$
and $\J$ is directly applicable.
Let's look at the more interesting $\alpha$ of type
\[
  \prod_{q : \Id_{A}(\fst(x),\fst(y)) \times \Id_B(\snd(x),\snd(y))} f(g(q)) = q.
\]
\begin{task}
  Implement $\alpha$.
  You may assume that $\refl(m) \concat p \jdeq p$ and $\ap_f(\refl(m)) \jdeq \refl(f(m))$,
  as usual.
  \begin{hint}
    Generalize the theorem
    so that you have free endpoints instead of $\fst(x)$ or $\snd(y)$.
  \end{hint}
\end{task}
\marksolution{
  \begin{solution}
    \begin{multline*}
      \textstyle
      \alpha \defeq \lambda q.\J[x_1.y_1.q_1.
        f(\ap^2_{\lambda ab.\pair{a}{b}}(q_1,\ap_\snd(q))) = \pair{q_1}{\ap_\snd(q)}
      ](\fst(q),z_1.
      \\
      J[x_2.y_2.q_2. f(\ap_{\lambda b.\pair{z_1}{b}}(q_2)) = \pair{\refl(z_1)}{q_2}]
      (\snd(q),
      z_2.\refl(\pair{\refl(z_1)}{\refl(z_2)})))
      \end{multline*}
  \end{solution}
}

\subsection{The Mysterious Case of Eckmann and Hilton}

As mentioned above, the equivalence between $\ap^2$
and $\ap'^2$ you gave in Task~\ref{task:ap2prime}
directly leads to the Eckmann-Hilton theorem,
\[
  \prod_{(m{:}A)}\prod_{(p,q{:}\refl_A(m)=\refl_A(m))}p\concat q = q\concat p,
\]
which asserts the abelianness of higher groups
induced by loops at $\refl_A(m)$ and path concatenation.
We will prove this in this subsection.

Note that this is the first ``big'' theorem
in homework assignments
and the code could easily become unreadable
if Favonia did not divide the theorem into multiple tasks.
The following optional text shows one way to write readable code
even when the proof is ridiculously long.
(You do not have to follow the style.)
\begin{mdframed}
The code of type $a = b$
often involves a long chain of transitivity from $a$ to $b$ through lots of intermediate points.
If we just write down paths $p_0 \concat (p_1 \concat \cdots p_{n-1} \cdots )$ as the code,
no one can understand the logic.
A better solution is to write down \emph{both} the proof \emph{and} the points
in a clear manner.
(In proof-irrelevant mathematics, one simply write down all the points as
$a_0 = a_1 = \cdots = a_n$.)
Here we will introduce to you the style adopted in the current \texttt{HoTT-Agda} library
(containing computer-checked HoTT proofs written for the proof assistant \texttt{Agda}).
The general form of this style is
\[
  \begin{array}{l}
    a_0
    \\
    \quad\reason{p_0}
    \\
    a_1
    \\
    \quad\vdots
    \\
    a_{n-1}
    \\
    \quad\reason{p_{n-1}}
    \\
    a_n
    \\
    \quad\qedsymbol
  \end{array}
\]
where $p_i$ is a path showing that $a_i$ and $a_{i+1}$ are equal.
For example, the following code snippet demonstrates the associativity among four paths,
assuming that the lemma $\concatassoc(p)(q)(r)$
inhabits the type $(p \concat q) \concat r = p \concat (q \concat r)$.
\[
  \begin{array}{l}
    ((p \concat q) \concat r) \concat s
    \\
    \quad\reason{\ap_{\lambda x.x\concat s}(\concatassoc(p)(q)(r))}
    \\
    (p \concat (q \concat r)) \concat s
    \\
    \quad\reason{\concatassoc(p)(q \concat r)(s)}
    \\
    p \concat ((q \concat r) \concat s)
    \\
    \quad\reason{\ap_{\lambda x.p\concat x}(\concatassoc(q)(r)(s))}
    \\
    p \concat (q \concat (r \concat s))
    \\
    \quad\qedsymbol
  \end{array}
\]
One can check that every path $p_i$ in the above example indeed has the type $a_i = a_{i+1}$.
If $a_i$ and $a_{i+1}$ are definitionally equal
but you nonetheless want to write them out for clarity,
a trivial path $\refl(a_i)$ can be used.
This style essentially marks the endpoints of every path
in the chain of transitivity,
but is clearer than using the primitive $\trans$ directly
(as you did in Homework~3).
As a side note,
the general form can be implemented as $p_0 \concat ( p_1 \concat \cdots (p_{n-1} \concat \refl(a_n)) \cdots)$,
and different implementations can give rise to different computational behaviors.
\end{mdframed}

\begin{task}
  Assuming $p, q : \refl(m) = \refl(m)$, show that
  \[
    \ap^2_{\lambda xy.x\concat y}(p,q) = p \concat q
  \]
  with access to the following lemmas.
  \begin{align*}
    \apid(p) &: \ap_{\id}(p) = p\\
    \lemmaone(p) &: \ap_{\lambda x.x\concat\refl(n)}(p) = p
  \end{align*}
  \begin{hint}
    $\lambda x.\refl(n)\concat x \jdeq \id$.
  \end{hint}
\end{task}
\markextra{
  \begin{mdframed}[style=extra]
    \textsc{Clarification \texttt{(2013/11/11 11am)}:}
    The $\mathtt{lemma}_1$ in Task 6 is taking a path of type
    $\mathtt{refl}(n) = \mathtt{refl}(n)$ for some $n$,
    not a path with generic endpoints,
    as this is what you need in proving the theorem.
    It is a good exercise to find out a generalized theorem
    which can be used directly as the motive of $\mathsf{J}$.
  \end{mdframed}
}
\marksolution{
  \begin{solution}
    $\displaystyle
      \mathtt{code}_1(p)(q) \defeq \ap^2_{\lambda xy.x\concat y}(\lemmaone(p),\apid(q))
    $
  \end{solution}
}

\begin{bonus}
  Similarly, show that $\ap'^2_{\lambda xy.x\concat y}(p,q) = q \concat p$.
\end{bonus}
\marksolution{
  \begin{solution}
    $\displaystyle
      \mathtt{code}_2(p)(q) \defeq \ap^2_{\lambda xy.x\concat y}(\apid(q),\lemmaone(q))
    $
  \end{solution}
}

\begin{task}
  Prove the Eckmann-Hilton theorem with the following lemma:
  \[
    \lemmatwo(f)(p)(q) : \ap^2_f(p,q) = \ap'^2_f(p,q).
  \]
\end{task}
\marksolution{
  \begin{solution}
    \[
      \lambda mpq.\mathtt{code}_1(p)(q)^{-1} \concat (\lemmatwo(\lambda pq.p \concat q)(p)(q) \concat \mathtt{code}_2(p)(q))
    \]
  \end{solution}
}

\section{Equivalence is an Equivalence}

Please finish the following tasks without using the Univalence Axiom.

\begin{task}
  \textbf{Reflexivity.} Show that $A \eqv A$ for any $A : \U$.
\end{task}
\marksolution{
  \begin{solution}
    $(f,(g,\alpha),(h,\beta))$ with the following components:
    \begin{align*}
      f &\defeq \id_A
      \\
      g &\defeq \id_A
      \\
      \alpha &\defeq \lambda a.\refl_A(a)
      \\
      h &\defeq \id_A
      \\
      \beta &\defeq \lambda a.\refl_A(a)
    \end{align*}
  \end{solution}
}

\begin{task}
  \textbf{Symmetry.}
  Assuming you have the data $(f,(g,\alpha),(h,\beta))$ of type $A \eqv B$, show that $B \eqv A$.
  You may assume that $\qinv(f)$ with the data $(g',\alpha',\beta')$ is also available,
  but it is fun to finish this task without using it.
\end{task}

\marksolution{
  \begin{solution}
    There are many ways without using $\qinv(f)$. This is not an exhaustive list.
    \begin{itemize}
      \item Picking $g$.
        \begin{align*}
          f' &\defeq g
          \\
          g' &\defeq f
          \\
          \alpha' &\defeq \lambda a. (\beta(g(f(a)))^{-1}) \concat (\ap_h(\alpha(f(a)))) \concat \beta(a))
          \\
          h' &\defeq f
          \\
          \beta' &\defeq \alpha
        \end{align*}
      \item Picking $h$.
        \begin{align*}
          f' &\defeq h
          \\
          g' &\defeq f
          \\
          \alpha' &\defeq \beta
          \\
          h' &\defeq f
          \\
          \beta' &\defeq \lambda b. (\alpha(f(g(b)))^{-1}) \concat (\ap_f(\beta(g(b))) \concat \alpha(b))
        \end{align*}
      \item Balanced choice.
        \begin{align*}
          f' &\defeq h \comp (f \comp g)
          \\
          g' &\defeq f
          \\
          \alpha' &\defeq \lambda a. \ap_h(\alpha(f(a))) \concat \beta(a)
          \\
          h' &\defeq f
          \\
          \beta' &\defeq \lambda b. \ap_f(\beta(g(b))) \concat \alpha(b)
        \end{align*}
    \end{itemize}
    Here's one way to use $\qinv(f)$.
    \begin{align*}
      f'' &\defeq g'
      \\
      g'' &\defeq f
      \\
      \alpha'' &\defeq \beta'
      \\
      h'' &\defeq f
      \\
      \beta'' &\defeq \alpha'
    \end{align*}
  \end{solution}
}

\begin{task}  
\textbf{Transitivity.}
  Assuming you have
  the data
  $(f_1,(g_1,\alpha_1),(h_1,\beta_1))$ of type $A \eqv B$
  and
  $(f_2,(g_2,\alpha_2),(h_2,\beta_2))$ of type $B \eqv C$
  at hand,
  show that $A \eqv C$.
\end{task}
\marksolution{
  \begin{solution}
    \begin{align*}
      f &\defeq f_2 \comp f_1
      \\
      g &\defeq g_1 \comp g_2
      \\
      \alpha &\defeq \lambda b. \ap_{f_2}(\alpha_1(g_2(b))) \concat \alpha_2(b)
      \\
      h &\defeq h_1 \comp h_2
      \\
      \beta &\defeq \lambda a. \ap_{h_1}(\beta_2(f_1(a))) \concat \beta_1(a)
    \end{align*}
  \end{solution}
}

\end{document}
