\documentclass[12pt]{article}

%%%%%%%%%%%%%%%%%%%%%%%%%%%%%%%%%%%%%%%%
% Basic packages
%%%%%%%%%%%%%%%%%%%%%%%%%%%%%%%%%%%%%%%%
\usepackage{amsmath,amsthm,amssymb}
\usepackage{mathtools}
\usepackage{etoolbox}
\usepackage{mathpartir}
\usepackage{verse}
\usepackage{hyperref}

%%%%%%%%%%%%%%%%%%%%%%%%%%%%%%%%%%%%%%%%
% Theorem-like environments
%%%%%%%%%%%%%%%%%%%%%%%%%%%%%%%%%%%%%%%%
\theoremstyle{plain}% default
\newtheorem{theorem}{Theorem}[section]
\newtheorem{lemma}{Lemma}[section]
\newtheorem{task}{Task}
\newtheorem{bonus}{Bonus Task}
\newenvironment{hint}{\noindent{\bf (Hint)}}{}

\theoremstyle{definition}
\newtheorem{definition}{Definition}[section]

\theoremstyle{remark}
\newtheorem*{remark}{Remark}
%%%%%%%%%%%%%%%%%%%%%%%%%%%%%%%%%%%%%%%%
% ITT + HoTT
%%%%%%%%%%%%%%%%%%%%%%%%%%%%%%%%%%%%%%%%
\newrobustcmd*{\ctx}{\Gamma}
\newrobustcmd*{\entails}{\vdash}
\newrobustcmd*{\comp}{\mathbin{\circ}}
\newrobustcmd*{\jdeq}{\equiv} %%% judgmental equality
\newrobustcmd*{\defeq}{\vcentcolon\equiv} %%% definition. Sorry I screwed this up.
\newrobustcmd*{\id}{\mathtt{id}}
\newrobustcmd*{\dummy}{\underline{\hspace{0.5em}}}
\newrobustcmd*{\unknown}{\text{--}}
\newrobustcmd*{\blank}{\underline{\hspace{2em}}}
% Units
\newrobustcmd*{\triv}{\langle\rangle}
%\newrobustcmd*{\triv}{\star} -- the star version
% Bottoms
\newrobustcmd*{\abort}{\mathtt{abort}}
% Products
\newrobustcmd*{\fst}{\mathtt{fst}}
\newrobustcmd*{\snd}{\mathtt{snd}}
\newrobustcmd*{\pair}[2]{\langle #1,#2\rangle}
\newrobustcmd*{\tuple}[2]{\langle #1,#2\rangle} % alternative
\newrobustcmd*{\tuplepair}[2]{\langle #1,#2\rangle} % alternative
% Sums
\newrobustcmd*{\inl}{\mathtt{inl}}
\newrobustcmd*{\inr}{\mathtt{inr}}
\newrobustcmd*{\case}{\mathtt{case}}
% Natural numbers
\newrobustcmd*{\rec}{\mathtt{rec}}
\newrobustcmd*{\zero}{\mathtt{zero}}
\newrobustcmd*{\suc}{\mathtt{suc}}
\newrobustcmd*{\N}{\mathbb{N}}
\newrobustcmd*{\nat}{\mathtt{nat}}
% Identities
\newrobustcmd*{\Id}{\mathtt{Id}}
\newrobustcmd*{\refl}{\mathtt{refl}}
\newrobustcmd*{\J}{\mathsf{J}}
\newrobustcmd*{\sym}{\mathtt{sym}}
\newrobustcmd*{\trans}{\mathtt{trans}}
\newrobustcmd*{\ap}{\mathtt{ap}}
\newrobustcmd*{\tr}{\mathtt{tr}}
%%% Path concatenation
\newrobustcmd*{\concat}{%
  \mathchoice{\mathbin{\raisebox{0.5ex}{$\displaystyle\centerdot$}}}%
  {\mathbin{\raisebox{0.5ex}{$\centerdot$}}}%
  {\mathbin{\raisebox{0.25ex}{$\scriptstyle\,\centerdot\,$}}}%
  {\mathbin{\raisebox{0.1ex}{$\scriptscriptstyle\,\centerdot\,$}}}
}
% Universes
\newrobustcmd*{\U}{\mathcal{U}}
% Equivalences
\newrobustcmd*{\eqv}{\simeq} % from HoTT-book
\newrobustcmd*{\qinv}{\mathsf{qinv}}
%%%%%%%%%%%%%%%%%%%%%%%%%%%%%%%%%%%%%%%%
% Names of lemmas
%%%%%%%%%%%%%%%%%%%%%%%%%%%%%%%%%%%%%%%%
\mathchardef\mhyphen="2D
\newrobustcmd*{\apconcat}{\mathtt{ap\mhyphen concat}}
\newrobustcmd*{\apid}{\mathtt{ap\mhyphen id}}
\newrobustcmd*{\apcomp}{\mathtt{ap\mhyphen comp}}
\newrobustcmd*{\lemmaone}{\mathtt{lemma}_1}
\newrobustcmd*{\lemmatwo}{\mathtt{lemma}_2}
%%%%%%%%%%%%%%%%%%%%%%%%%%%%%%%%%%%%%%%%
% Equational reasoning
%%%%%%%%%%%%%%%%%%%%%%%%%%%%%%%%%%%%%%%%
\newrobustcmd*{\reason}[1]{=\hspace{-0.5em}\langle\hspace{0.2em} #1 \hspace{0.2em}\rangle}
\newrobustcmd*{\concatassoc}{\mathtt{concat\mhyphen assoc}}


\title{\Large\textbf{
  Homework 4: Second Identity Crisis}
\normalsize\\
15-819 Homotopy Type Theory\\
Favonia (\href{mailto:favonia@cmu.edu}{favonia@cmu.edu})}

\author{}

\date{%
Out: 4/Nov/13\\
Due: 18/Nov/13
}

\begin{document}

\maketitle

\section{Application Programming Interface}

Assume
$\ctx \entails f : A \to B$ and
$\ctx \entails g : B \to C$.

\begin{task}
  Implement $\apconcat$ and $\apcomp$.
  \begin{align*}
    &\ctx \entails \apconcat : \prod_{l,m,n:A}\prod_{p:l=m}\prod_{q:m=n}\ap_f(p \concat q) = \ap_f(p) \concat \ap_f(q)
    \\
    &\ctx \entails \apcomp : \prod_{m,n:A}\prod_{p:m=n}\ap_{g \comp f}(p) = \ap_g(\ap_f(p))
  \end{align*}
  You may assume that $\refl(m) \concat p \jdeq p$, $\refl(m)^{-1} \jdeq \refl(m)$,
  and $\ap_f(\refl(m)) \jdeq \refl(f(m))$.
  \begin{hint}
    Make sure that your motives are well-typed
    as you did for the associativity in Homework~3.
  \end{hint}
\end{task}

\section{Paths to Infinity}

Show that
\[
  \prod_{x,y:A} \Id_A(x,y) \eqv F(x,y)
\]
for some ``explicit'' $F$.
Recall that the data format of an equivalence is
$(f,(g,\alpha),(h,\beta))$
where $\alpha$ is a witness of $g$ being a right-inverse of $f$
and $\beta$ is of $h$ being a left-inverse.

\subsection{Tops and Bottoms are Dual in Many Ways}

\begin{task}
  Show that $\prod_{x,y:\top} \Id_\top(x,y) \eqv \top$
  (where $F(x,y) \defeq \top$).
  You do not have to justify your code.
  \begin{hint}
    What is the $\eta$ rule of $\top$?
  \end{hint}
\end{task}

\begin{task}
  Show that $\prod_{x,y:\bot} \Id_\bot(x,y) \eqv \bot$
  (where $F(x,y) \defeq \bot$).
  You do not have to justify your code.
  \begin{hint}
    Favonia can answer this with less than 30 characters.
  \end{hint}
\end{task}

\subsection{Pathbreaking Techniques}

Assume that $f : C \to D \to E$,
$p : c_1 =_C c_2$, and $q : d_1 =_D d_2$
in the current context.
\[
  \ap^2_f(p,q) \defeq \ap_{\lambda c.f(c)(d_1)}(p) \concat \ap_{f(c_2)}(q)
\]
\begin{task}
  \label{task:ap2prime}
  Find another one which is ``symmetric'' to the above implementation,
  where $q$ comes before $p$ instead of $p$ coming before $q$.
  (We will refer to this implementation as $\ap'^2_f$.)
\end{task}

\[
  \begin{array}{lcl}
    f &\defeq& \lambda p.\pair{\ap_\fst(p)}{\ap_\snd(p)}
    \\
    g &\defeq& \lambda q.\ap^2_{\lambda ab.\pair{a}{b}}(\fst(q),\snd(q))
    \\
    h &\defeq& g
  \end{array}
\]
Let's look at the more interesting $\alpha$ of type
\[
  \prod_{q : \Id_{A}(\fst(x),\fst(y)) \times \Id_B(\snd(x),\snd(y))} f(g(q)) = q.
\]
\begin{task}
  Implement $\alpha$.
  You may assume that $\refl(m) \concat p \jdeq p$ and $\ap_f(\refl(m)) \jdeq \refl(f(m))$,
  as usual.
  \begin{hint}
    Generalize the theorem
    so that you have free endpoints instead of $\fst(x)$ or $\snd(y)$.
  \end{hint}
\end{task}

\subsection{The Mysterious Case of Eckmann and Hilton}

\[
  \prod_{(m{:}A)}\prod_{(p,q{:}\refl_A(m)=\refl_A(m))}p\concat q = q\concat p,
\]

\begin{task}
  Assuming $p, q : \refl(m) = \refl(m)$, show that
  \[
    \ap^2_{\lambda xy.x\concat y}(p,q) = p \concat q
  \]
  with access to the following lemmas.
  \begin{align*}
    \apid(p) &: \ap_{\id}(p) = p\\
    \lemmaone(p) &: \ap_{\lambda x.x\concat\refl(n)}(p) = p
  \end{align*}
  \begin{hint}
    $\lambda x.\refl(n)\concat x \jdeq \id$.
  \end{hint}
\end{task}

\begin{bonus}
  Similarly, show that $\ap'^2_{\lambda xy.x\concat y}(p,q) = q \concat p$.
\end{bonus}

\begin{task}
  Prove the Eckmann-Hilton theorem with the following lemma:
  \[
    \lemmatwo(f)(p)(q) : \ap^2_f(p,q) = \ap'^2_f(p,q).
  \]
\end{task}

\section{Equivalence is an Equivalence}

Please finish the following tasks without using the Univalence Axiom.

\begin{task}
  \textbf{Reflexivity.} Show that $A \eqv A$ for any $A : \U$.
\end{task}

\begin{task}
  \textbf{Symmetry.}
  Assuming you have the data $(f,(g,\alpha),(h,\beta))$ of type $A \eqv B$, show that $B \eqv A$.
  You may assume that $\qinv(f)$ with the data $(g',\alpha',\beta')$ is also available,
  but it is fun to finish this task without using it.
\end{task}

\begin{task}  
\textbf{Transitivity.}
  Assuming you have
  the data
  $(f_1,(g_1,\alpha_1),(h_1,\beta_1))$ of type $A \eqv B$
  and
  $(f_2,(g_2,\alpha_2),(h_2,\beta_2))$ of type $B \eqv C$
  at hand,
  show that $A \eqv C$.
\end{task}

\end{document}
