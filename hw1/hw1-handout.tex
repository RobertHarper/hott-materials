\documentclass[12pt]{article}

%%%%%%%%%%%%%%%%%%%%%%%%%%%%%%%%%%%%%%%%
% Basic packages
%%%%%%%%%%%%%%%%%%%%%%%%%%%%%%%%%%%%%%%%
\usepackage{amsmath,amsthm,amssymb}
\usepackage{mathpartir}
\usepackage[usenames,svgnames]{xcolor}
\usepackage{hyperref}
\usepackage{xspace}

%%%%%%%%%%%%%%%%%%%%%%%%%%%%%%%%%%%%%%%%
% Theorem-like environments
%%%%%%%%%%%%%%%%%%%%%%%%%%%%%%%%%%%%%%%%
\newtheorem{lemma}{Lemma}[section]
\newenvironment{hint}{\noindent{\bf (Hint)}}{}
\newtheorem{thm}{Theorem}
\newtheorem{definition}{Definition}
\newtheorem{task}{Task}
\newtheorem{bonus}{Bonus Task}
\renewenvironment{proof}{\trivlist \item[\hskip \labelsep{\bf
Proof:}]}{\hfill$\Box$ \endtrivlist}

%%%%%%%%%%%%%%%%%%%%%%%%%%%%%%%%%%%%%%%%
% Macros for Heyting Algebra
%%%%%%%%%%%%%%%%%%%%%%%%%%%%%%%%%%%%%%%%
\renewcommand*{\equiv}{\simeq}
\newcommand*{\init}{\bot}
\newcommand*{\final}{\top}
\newcommand*{\meet}{\wedge}
\newcommand*{\join}{\vee}
\renewcommand*{\exp}{\supset}

%%%%%%%%%%%%%%%%%%%%%%%%%%%%%%%%%%%%%%%%
% Boolean Algebra
%%%%%%%%%%%%%%%%%%%%%%%%%%%%%%%%%%%%%%%%
\newcommand*{\comp}[1]{\overline{#1}}

%%%%%%%%%%%%%%%%%%%%%%%%%%%%%%%%%%%%%%%%
% IPL
%%%%%%%%%%%%%%%%%%%%%%%%%%%%%%%%%%%%%%%%
\newcommand*{\ctx}{\Gamma}
\newcommand*{\entails}{\vdash}

\newcommand*{\judgmentfont}[1]{\textsf{#1}}
\newcommand*{\postfixjudgment}[1]{%
  \relax\ifnum\lastnodetype>0\mskip\medmuskip\fi
  \text{\judgmentfont{#1}}%
}
\newcommand*{\prop}{\postfixjudgment{prop}}
\newcommand*{\true}{\postfixjudgment{true}}

\newcommand*{\conj}{\wedge}
\newcommand*{\disj}{\vee}
\newcommand*{\imp}{\supset}

\usepackage{palatino}
\usepackage{mdframed}

\newcommand{\cut}[1]{}
\def\StripPrefix#1>{}
\def\jobis#1{FF\fi
  \def\predicate{#1}%
  \edef\predicate{\expandafter\StripPrefix\meaning\predicate}%
  \edef\job{\jobname}%
  \ifx\job\predicate
}
\if\jobis{hw1-solution}%
  \newcommand{\showsol}[1]{\color{FireBrick}#1\normalcolor}%
\else%
  \newcommand{\showsol}[1]{\cut{#1}}%
\fi
\newenvironment{sol}{\trivlist \item[\hskip \labelsep{\bf
Solution:}]}{\endtrivlist}
\newcommand{\showextra}[1]{\color{DarkOliveGreen}#1\normalcolor}
\mdfdefinestyle{extra}{linecolor=DarkOliveGreen,fontcolor=DarkOliveGreen}

\title{\Large\textbf{
  Homework 1: Heyting Algebra and IPL}
\normalsize\\
15-819 Homotopy Type Theory\\
Favonia (\href{mailto:favonia@cs.cmu.edu}{favonia@cs.cmu.edu})}

\author{}

\date{%
Out: 19/Sep/13\\
Due: 3/Oct/13
}

\begin{document}

\maketitle

\showextra{
  \begin{mdframed}[style=extra]
    \textsc{Change \texttt{(2013/10/1 10pm)}:} There was a 24-hour extension.
    The new deadline is shown above.
  \end{mdframed}
}

This is 15-819's first homework assignment!

\section{Heyting Meets Boole}

The goal of this section is to reason about an algebra,
for example a Boolean algebra,
through abstract properties,
without appealing a particular model,
such as \texttt{true} and \texttt{false}.
In particular, meets and joins are defined by their \emph{universal properties}
instead of truth tables.

As mentioned in the lectures, with implications one can show distributiveness
of any Heyting algebra.
The following is one of the most interesting parts in the proof.
\begin{task}
  Show that $A \meet (B \join C) \leq (A \meet B) \join (A \meet C)$
  in any Heyting algebra.

  \begin{hint}
    You might find Yoneda's Lemma useful, which says (in this particular context)
    $A \leq B$ iff for all $C$, $B \leq C$ implies $A \leq C$.
    There is a short proof with Yoneda's, and another short proof without.
  \end{hint}
\end{task}

\showsol{
\begin{sol}
  (This is a demonstration of derivations.)
  {
    \scriptsize
    \[
      \infer{
        \infer{ }{A \meet (B \join C) \leq B \join C}
        \\
        \infer{
          \infer{
            \infer{ }{A \meet B \leq (A \meet B) \join (A \meet C)}
          }{
            B \leq A \exp (A \meet B) \join (A \meet C)
          }
          \\
          \infer{
            \infer{ }{A \meet C \leq (A \meet B) \join (A \meet C)}
          }{
            C \leq A \exp (A \meet B) \join (A \meet C)
          }
        }{
          B \join C \leq A \exp (A \meet B) \join (A \meet C)
        }
      }{
        A \meet (B \join C) \leq A \exp (A \meet B) \join (A \meet C)
      }
    \]
  }
  and so
  {\scriptsize
    \[
      \infer{
        \infer{ }{A \meet (B \join C) \leq A}
        \\
        A \meet (B \join C) \leq A \exp (A \meet B) \join (A \meet C)
      }{
        A \meet (B \join C) \leq A \meet (A \exp (A \meet B) \join (A \meet C))
      }
    \]
  }
  and finally
  {\scriptsize
    \[
      \infer{
        A \meet (B \join C) \leq A \meet (A \exp (A \meet B) \join (A \meet C))
        \\
        \infer{ }{A \meet (A \exp (A \meet B) \join (A \meet C)) \leq (A \meet B) \join (A \meet C)}
      }{
        A \meet (B \join C) \leq (A \meet B) \join (A \meet C)
      }.
    \]
  }
\end{sol}
}

In class we also gave two definitions of negations $\neg A$,
one with explicit construction and the other through universal properties.
The next task is to show that these two definitions are equivalent.
\begin{task}
  Show that in any Heyting algebra,
  $A \exp \init$ is one of the largest elements inconsistent with $A$,
  and is equivalent to any largest inconsistent one.
\end{task}
\showextra{
  \begin{mdframed}[style=extra]
    \textsc{Change \texttt{(2013/9/26 4pm)}:}
    The above task has been reworded for clarification.
  \end{mdframed}
}
\showsol{
\begin{sol}
  By definition we have $A \meet (A \exp \init) \leq \init$ and also for any $B$,
  \[
    \infer{
      A \meet B \leq \init
    }{
      B \leq A \exp \init
    }.
  \]
  All largest such elements are isomorphic simply because of being largest.
\end{sol}
}

Finally, with the introduction of mighty complements,
exponentials become definable if distributiveness is assumed.
As a corollary, in Boolean algebras negations and complements collide.
(Please refer to the lecture note for the correct definition of complements.
There was a mistake in the definition given in class.)
\begin{task}
  Show that in any Boolean algebra (complemented distributive lattice),
  $\comp A \join B$ is a valid implementation of $A \exp B$.
  That is, it satisfies all properties of $A \exp B$.
\end{task}
\showsol{
\begin{sol}\mbox{}
  \begin{itemize}
    \item ``$A \meet (A \exp B) \leq B$.''

      We have $A \meet (\comp A \join B) \leq (A \meet \comp A) \join (A \meet B)$.
      For the first branch, $A \meet \comp A \leq \init \leq B$.
      For the second branch, $A \meet B \leq B$.
      Thus $A \meet (\comp A \join B) \leq B$.
    \item ``If $A \meet C \leq B$ then $C \leq A \exp B$.''

      Since $C \leq \final \leq A \join \comp A$,
      $C \leq (A \join \comp A) \meet C \leq (A \meet C) \join (\comp A \meet C)$.
      For the first branch, $A \meet C \leq B \leq \comp A \join B$.
      For the second branch, $\comp A \meet C \leq \comp A \leq \comp A \join B$.
      Therefore $C \leq \comp A \join B$.
  \end{itemize}
\end{sol}
}

\section{IPL Structural Engineering}

Here we will explore structural properties of IPL, among which
one of the most important is \emph{transitive} shown below.

\begin{task}
  Show that IPL is transitive, which is to say
  \[
    \infer{
      \ctx, \ctx' \entails P \true
      \\
      \ctx, P \true, \ctx' \entails Q \true
    }{
      \ctx, \ctx' \entails Q \true
    }
  \]
  is admissible.
  You only have to consider the case that the last rule applied in the right sub-derivation
  (of $\ctx, P \true, \ctx' \entails Q \true$)
  is either the primitive reflexivity or rules in the negative fragment.
  You may assume weakening and exchange as admissible rules.
\end{task}
\showsol{
\begin{sol}
  Induction on the right sub-derivation. Consider the last rule applied.
  \begin{itemize}
    \item Reflexivity.

      If $Q \true$ is proved from that particular $P \true$ in the context by reflexivity,
      then $Q \true = P \true$ and by the assumption $\ctx, \ctx' \entails P \true$.
      Otherwise, that particular $P \true$ is irrelevant
      and one can apply reflexivity to the rest of the context to obtain $Q \true$.

    \item $\top$I.
      \[
        \infer{
        }{
          \ctx, P \true, \ctx' \entails \top \true
        }.
      \]
      By the same rule $\ctx, \ctx' \entails \top \true$.
    
    \item $\meet$I.
      \[
        \infer{
          \ctx, P \true, \ctx' \entails Q \true
          \\
          \ctx, P \true, \ctx' \entails R \true
        }{
          \ctx, P \true, \ctx' \entails Q \meet R \true
        }.
      \]
      By inductive hypotheses we have
      $\ctx, \ctx' \entails Q \true$
      and
      $\ctx, \ctx' \entails R \true$.
      Therefore
      \[
        \infer{
          \ctx, \ctx' \entails Q \true
          \\
          \ctx, \ctx' \entails R \true
        }{
          \ctx, \ctx' \entails Q \meet R \true
        }.
      \]
    \item $\meet$E$_1$.
      \[
        \infer{
          \ctx, P \true, \ctx' \entails Q \meet R \true
        }{
          \ctx, P \true, \ctx' \entails Q \true
        }.
      \]
      By inductive hypothesis $\ctx, \ctx' \entails Q \meet R \true$.
      And so $\ctx, \ctx' \entails Q \true$.
    \item $\meet$E$_2$.
      \[
        \infer{
          \ctx, P \true, \ctx' \entails R \meet Q \true
        }{
          \ctx, P \true, \ctx' \entails Q \true
        }.
      \]
      By inductive hypothesis $\ctx, \ctx' \entails R \meet Q \true$.
      And so $\ctx, \ctx' \entails Q \true$.
    \item $\imp$I.
      \[
        \infer{
          \ctx, P \true, \ctx', Q \true \entails R \true
        }{
          \ctx, P \true, \ctx' \entails Q \imp R \true
        }.
      \]
      By weakening $\ctx, \ctx', Q \true \entails P \true$.
      By inductive hypothesis $\ctx, \ctx', Q \true \entails R \true$
      and thus $\ctx, \ctx' \entails Q \imp R \true$.
    \item $\imp$E.
      \[
        \infer{
          \ctx, P \true, \ctx' \entails R \imp Q \true
          \\
          \ctx, P \true, \ctx' \entails R \true
        }{
          \ctx, P \true, \ctx' \entails Q \true
        }.
      \]
      By inductive hypotheses
      $\ctx, \ctx' \entails R \imp Q \true$
      and
      $\ctx, \ctx' \entails R \true$.
      Thus from the same rule
      $\ctx, \ctx' \entails Q \true$.
  \end{itemize}
\end{sol}
}

\section{Semantical Analysis of IPL}

Any Heyting algebra can be a model of IPL.
In fact, $\ctx \entails P \true$ is provable iff $\ctx^+ \leq P^*$ in any Heyting algebra,
where $(\text{--})^*$ is the straightforward lifting of
any evaluation function from atomic propositions
to elements in the Heyting algebra in question,
and $\ctx^+$ is defined as the comprehension of $(\text{--})^*$
through the following equations:
\begin{enumerate}
  \item $\cdot^+ = \top$.
  \item $(\ctx, P\true)^+ = P^* \meet \ctx^+$.
\end{enumerate}

\begin{task}
  Show that for any Heyting algebra and any evaluation function on atoms,
  if $\ctx \entails P \true$ then $\ctx^+ \leq P^*$.
  You only have to consider the cases in which the last rule applied is ($\imp$I) or ($\imp$E).
\end{task}

\showextra{
  \begin{mdframed}[style=extra]
    \textsc{Change \texttt{(2013/10/1 10pm)}:}
    The task is intended to be a general statement about any Heyting algebra
    and any evaluation function on atoms.  The evaluation function on propositions
    is always lifted (in the way described in class) from the evaluation function on atoms.
  \end{mdframed}
}

\showsol{
\begin{sol}\mbox{}
  \begin{itemize}
    \item $\imp$I.

      By inductive hypothesis we have
      $P^* \meet \ctx^+ \leq Q^*$
      and thus $\ctx^+ \leq P^* \exp Q^* = (P \exp Q)^*$.
    \item $\imp$E.

      By inductive hypotheses $\ctx^+ \leq P^* \exp Q^*$ and $\ctx^+ \leq P^*$.
      Therefore $\ctx^+ \leq P^* \meet (P^* \exp Q^*) \leq Q^*$.
  \end{itemize}
\end{sol}
}

\begin{task}
  Consider the Lindenbaum algebra of IPL where
  the elements are all propositions in IPL
  (with the translation $(\text{--})^*$ being the identity function)
  and the relationship $\leq$ is defined by provability in IPL.%
  \footnote{To simplify the problem, we avoid taking quotients by interprovability,
  but one must consider that if a partial order is desired.}
  That is, $A \leq B$ iff $A \true \entails B \true$.
  Show that this is a Heyting algebra.
  You only have to prove the transitivity.
  You may assume weakening and exchange of IPL, or cite previous tasks as lemmas.
\end{task}

\showsol{
  \begin{sol}\mbox{}
    \begin{itemize}
      \item Transitivity.

        Suppose $A \leq C$ and $C \leq B$.
        By weakening of IPL, $C \true, A \true \entails B \true$.
        By transitivity of IPL, $A \true \entails B \true$, and thus $A \leq B$.
    \end{itemize}
  \end{sol}
}

\showextra{
  \begin{mdframed}[style=extra]
    \textsc{Remark \texttt{(2013/9/25 9am)}:}
    To really complete the theorem, you need to show that $\Gamma \entails A \true$
    iff $\Gamma^+ \true \entails A \true$. You do not have to prove this
    for this homework.
  \end{mdframed}
}

\end{document}
