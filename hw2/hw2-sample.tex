\documentclass[12pt]{article}

%%%%%%%%%%%%%%%%%%%%%%%%%%%%%%%%%%%%%%%%
% Basic packages
%%%%%%%%%%%%%%%%%%%%%%%%%%%%%%%%%%%%%%%%
\usepackage{amsmath,amsthm,amssymb}
\usepackage{mathpartir}
\usepackage{hyperref}
\usepackage{stmaryrd}
\usepackage[all]{xy}

%%%%%%%%%%%%%%%%%%%%%%%%%%%%%%%%%%%%%%%%
% Theorem-like environments
%%%%%%%%%%%%%%%%%%%%%%%%%%%%%%%%%%%%%%%%
\newtheorem{lemma}{Lemma}[section]
\newenvironment{hint}{\noindent{\bf (Hint)}}{}
\newtheorem{thm}{Theorem}
\newtheorem*{remark}{Remark}
\newtheorem{definition}{Definition}
\newtheorem{task}{Task}
\newtheorem{bonus}{Bonus Task}
\renewenvironment{proof}{\trivlist \item[\hskip \labelsep{\bf
Proof:}]}{\hfill$\Box$ \endtrivlist}

%%%%%%%%%%%%%%%%%%%%%%%%%%%%%%%%%%%%%%%%
% BiCCC
%%%%%%%%%%%%%%%%%%%%%%%%%%%%%%%%%%%%%%%%
\newcommand*{\sem}[1]{\llbracket #1\rrbracket} % sem = semantics
\newcommand*{\id}{\mathtt{id}}
\newcommand*{\comp}{\mathbin{\circ}}
\newcommand*{\pair}[2]{\langle #1,#2\rangle}
\newcommand*{\fst}{\mathtt{fst}}
\newcommand*{\snd}{\mathtt{snd}}
\newcommand*{\copair}[2]{\{ #1,#2\}} % e.g. coconuts = nuts
\newcommand*{\inl}{\mathtt{inl}}
\newcommand*{\inr}{\mathtt{inr}}
\newcommand*{\ap}{\mathtt{ap}}

%%%%%%%%%%%%%%%%%%%%%%%%%%%%%%%%%%%%%%%%
% IPL
%%%%%%%%%%%%%%%%%%%%%%%%%%%%%%%%%%%%%%%%
\newcommand*{\ctx}{\Gamma}
\newcommand*{\entails}{\vdash}
\newcommand*{\defeq}{\equiv} % definitional equality
% case
\newcommand*{\case}{\mathtt{case}}
% NNO and natural numbers
\newcommand*{\rec}{\mathtt{rec}}
\newcommand*{\zero}{\mathtt{zero}}
\newcommand*{\suc}{\mathtt{suc}}
\newcommand*{\N}{\mathbb{N}}
\newcommand*{\nat}{\mathtt{nat}}
\newcommand*{\altrec}{\widetilde{\mathtt{rec}}} % alt = alternative


% I don't think everyone likes this font.
%\usepackage{palatino}

\title{\Large\textbf{
  Homework 2: Kindom of Kittens}
\normalsize\\
15-819 Homotopy Type Theory}

\author{}

\date{%
Out: 5/Oct/13\\
Due: 19/Oct/13
}

\begin{document}

\maketitle

\section{Categorical Considerations}

\subsection{Structural Safety}

\begin{task}\label{task:struct}
  Write down a suitable morphism in terms of the primitive constructs
  and the morphisms immediately available in each subtask.
  The primitive constructs include
  $\id$, $f \comp g$, $\pair{f}{g}$, $\fst$, $\snd$, $\inl$, $\inr$, $\copair{f}{g}$, $\lambda(f)$ and $\ap$.
  \begin{itemize}
    \item
      \textbf{Reflexivity}.

      Write down a morphism from $\Gamma^+ \times P^*$ to $P^*$.
    \item
      \textbf{Contraction}.

      Write down a morphism from $\Gamma^+ \times P^*$ to $Q^*$
      in terms of a morphism
      $f : (\Gamma^+ \times P^*) \times P^* \to Q^*$.
    \item
      \textbf{Weakening}.

      Write down a morphism from $\Gamma^+ \times P^*$ to $Q^*$
      in terms of a morphism
      $f : \Gamma^+ \to Q^*$.
    \item
      \textbf{Exchange}.

      Write down a morphism from
      $(\Gamma^+ \times Q^*) \times P^*$ to $R^*$
      in terms of a morphism
      $f : (\Gamma^+ \times P^*) \times Q^* \to R^*$.
    \item
      \textbf{Substitution}.

      Write down a morphism from $\Gamma^+$ to $Q^*$ in terms of
      two morphisms
      $f : \Gamma^+ \to P^*$
      and
      $g : \Gamma^+ \times P^* \to Q^*$.
  \end{itemize}
\end{task}

\subsection{Utility of Universality}

\begin{task}
  Show these rules via universal properties:
  \begin{itemize}
    \item $\pair{f}{g} \comp h = \pair{f\comp h}{g\comp h}$
    \item $h \comp \copair{f}{g} = \copair{h\comp f}{h\comp g}$
  \end{itemize}
\end{task}

\begin{task}
  Show that $\beta$ and $\eta$ rules of implications (function types)
  are justified by the universal property of exponentials.
  You may assume the translation of judgments $\sem{\text{--}}$
  satisfies the following properties:
  \begin{itemize}
    \item $\sem{\Gamma \entails \lambda x.M_1 : P \to Q}
      = \lambda (\sem{\Gamma, x{:}P \entails M_1 : Q})$
    \item $\sem{\Gamma \entails M_1\,M_2 : Q}
      = \ap \comp \pair{\sem{\Gamma \entails M_1 : P \to Q}}{\sem{\Gamma \entails M_2 : P}}$
    \item
      $\sem{\Gamma, x{:}P \entails x : P}$ gives the same morphism
      as you did in Task~\ref{task:struct} for reflexivity.
    \item
      $\sem{\Gamma, x{:}P \entails M : Q}$ also gives the same morphism
      in terms of $\sem{\Gamma \entails M : Q}$
      as you did in Task~\ref{task:struct} for weakening.
    \item
      Fortunately,%
      \footnote{These are all provable theorems for a suitable, systematic translation.}
      $\sem{\Gamma \entails [M_2/x]M_1 : Q}$ gives the same morphism
      in terms of $\sem{\Gamma, x{:}P \entails M_1 : Q}$ and $\sem{\Gamma \entails M_2 : P}$
      as you did in Task~\ref{task:struct} for substitution.
  \end{itemize}
  You may freely assume previous tasks, as usual.
\end{task}

\section{Commuting Conversion}

\begin{bonus}
  There is a short proof of
  $[M_1/x]N \defeq [M_2/x]N$ when $M_1 \defeq M_2$
  directly from suitable $\beta\eta$ rules and congruence.
  Try to find it!
\end{bonus}

\begin{task}
  Check that
  \[
    M \defeq \case(z;x.[\inl(x)/z]M;y.[\inr(y)/z]M)
  \]
  is derivable from
  \begin{itemize}
    \item $z \defeq \case(z;x.\inl(x);y.\inr(y))$ and
    \item
      $[\case(z;x.\inl(x);y.\inr(y))/z]M \equiv \case(z;x.[\inl(x)/z]M;y.[\inr(y)/z]M)$.
      
      (This is called commuting conversion.)
  \end{itemize}
  You may assume $[z/z]M \defeq M$ and previous tasks.
\end{task}

\section{Narratives of Natural Numbers}

\subsection{Summary in Sum}

Recall the alternative definition of natural number objects
through this diagram:
\[
  \xymatrixcolsep{5pc}
  \xymatrix{
    1 + \mathbb{N} \ar@{.>}[r]_{\id + \rec_{f,g}} \ar[d]_{\copair{\zero}{\suc}}
    & 1 + A \ar[d]^{\copair{f}{g}}
    \\
    \mathbb{N} \ar@{.>}[r]_{!\,\rec_{f,g}} & A
  }
\]
\begin{task}
  Check that this diagram indeed defines a natural number object
  (in the sense described in class).
\end{task}

\subsection{Recursors Redefined}

Here is the formal specification:
\begin{itemize}
  \item $\altrec(P; x.y.Q)(\zero) \defeq P$
  \item For any closed term $\emptyset \entails M : \nat$,\\
    $\altrec(P; x.y.Q)(\suc(M)) \defeq [M,\altrec(P;x.y.Q)(M)/x,y]Q$
\end{itemize}
Note that $Q$ has access to the \emph{predecessor} of the argument,
not the argument itself.

\begin{task}
  Define $\altrec$ in terms of $\rec$.
  You do not have to show the correctness.
\end{task}

\end{document}
