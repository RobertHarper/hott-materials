\documentclass[12pt]{article}

%%%%%%%%%%%%%%%%%%%%%%%%%%%%%%%%%%%%%%%%
% Basic packages
%%%%%%%%%%%%%%%%%%%%%%%%%%%%%%%%%%%%%%%%
\usepackage{amsmath,amsthm,amssymb}
\usepackage{mathtools}
\usepackage{etoolbox}
\usepackage{mathpartir}
\usepackage{hyperref}
\usepackage{stmaryrd}
\usepackage[all]{xy}

%%%%%%%%%%%%%%%%%%%%%%%%%%%%%%%%%%%%%%%%
% Theorem-like environments
%%%%%%%%%%%%%%%%%%%%%%%%%%%%%%%%%%%%%%%%
\theoremstyle{plain}% default
\newtheorem{theorem}{Theorem}[section]
\newtheorem{lemma}{Lemma}[section]
\newtheorem{task}{Task}
\newtheorem{bonus}{Bonus Task}
\newenvironment{hint}{\noindent{\bf (Hint)}}{}

\theoremstyle{definition}
\newtheorem{definition}{Definition}[section]

\theoremstyle{remark}
\newtheorem*{remark}{Remark}
%%%%%%%%%%%%%%%%%%%%%%%%%%%%%%%%%%%%%%%%
% ITT
%%%%%%%%%%%%%%%%%%%%%%%%%%%%%%%%%%%%%%%%
\newrobustcmd*{\ctx}{\Gamma}
\newrobustcmd*{\entails}{\vdash}
\newrobustcmd*{\jdeq}{\equiv} %%% judgmental equality
\newrobustcmd*{\defeq}{\vcentcolon\equiv} %%% definition. Sorry I screwed this up.
\newrobustcmd*{\id}{\mathtt{id}}
\newrobustcmd*{\dummy}{\underline{\hspace{0.5em}}}
% Units
\newrobustcmd*{\triv}{\langle\rangle}
% Products
\newrobustcmd*{\fst}{\mathtt{fst}}
\newrobustcmd*{\snd}{\mathtt{snd}}
\newrobustcmd*{\pair}[2]{\langle #1,#2\rangle}
\newrobustcmd*{\tuplepair}[2]{\langle #1,#2\rangle}
% Sums
\newrobustcmd*{\inl}{\mathtt{inl}}
\newrobustcmd*{\inr}{\mathtt{inr}}
\newrobustcmd*{\case}{\mathtt{case}}
% Natural numbers
\newrobustcmd*{\rec}{\mathtt{rec}}
\newrobustcmd*{\zero}{\mathtt{zero}}
\newrobustcmd*{\suc}{\mathtt{suc}}
\newrobustcmd*{\N}{\mathbb{N}}
\newrobustcmd*{\nat}{\mathtt{nat}}
% Identity
\newrobustcmd*{\Id}{\mathtt{Id}}
\newrobustcmd*{\refl}{\mathtt{refl}}
\newrobustcmd*{\J}{\mathsf{J}}
\newrobustcmd*{\sym}{\mathtt{sym}}
\newrobustcmd*{\trans}{\mathtt{trans}}
\newrobustcmd*{\ap}{\mathtt{ap}}
\newrobustcmd*{\tr}{\mathtt{tr}}
%%% Path concatenation
\newrobustcmd*{\concat}{%
  \mathchoice{\mathbin{\raisebox{0.5ex}{$\displaystyle\centerdot$}}}%
  {\mathbin{\raisebox{0.5ex}{$\centerdot$}}}%
  {\mathbin{\raisebox{0.25ex}{$\scriptstyle\,\centerdot\,$}}}%
  {\mathbin{\raisebox{0.1ex}{$\scriptscriptstyle\,\centerdot\,$}}}
}
% Universes
\newrobustcmd*{\U}{\mathcal{U}}


\title{\Large\textbf{
  Homework 3: Identity Crisis}
\normalsize\\
15-819 Homotopy Type Theory\\
Favonia (\href{mailto:favonia@cmu.edu}{favonia@cmu.edu})}

\author{}

\date{%
Out: 19/Oct/13\\
Due: 2/Nov/13
}

\begin{document}

\maketitle

\section{J Walkers}

\subsection{Transitive Transportation}

\begin{task}
  Implement these programs.
  \begin{enumerate}
    \item $\trans'_A$ by induction on another path argument.
    \item $\trans''_A$ in terms of $\tr$.
    \item
      $\ap'_f(p)$ in terms of $\tr$,
      where (in the current context) $f : A \to B$
      and $p : \Id_A(m,n)$.
  \end{enumerate}
  In addition, briefly summarize the computational rules
  (definitional equalities)
  your implementation satisfies.
  For example, the $\trans_A$ given in class
  satisfies the following definitional equality.
  \[
    \trans_A(m)(m)(n)(\refl_A(m))(p) \jdeq p
  \]
\end{task}

\subsection{Group Ticket}

\[
  \prod_{(m{:}A)}\prod_{(p,q{:}\refl_A(m)=\refl_A(m))}p\concat q = q\concat p,
\]
\begin{bonus}
Briefly explain what would fail if you took the theorem as the motive of $\J$ directly.
\end{bonus}

\begin{task}
  Write code with the following types.
  You may assume that
  \[
    \trans_A(m)(m)(n)(\refl_A(m))(p) \jdeq p
  \]
  and
  \[
    \sym_A(m)(m)(\refl_A(m)) \jdeq \refl_A(m).
  \]
  Please write down the dependency on endpoints in the motive of $\J$ explicitly.
  \begin{itemize}
    \item
      \textbf{Inverse-right.}
      ``$p \concat p^{-1} = \refl$''.
      \begin{multline*}
        \textstyle
        \prod_{m{:}A}\prod_{n{:}A}\prod_{p{:}\Id_A(m,n)}\\
          \Id_{\Id_A(m,m)}(\trans_A(m)(n)(m)(p)(\sym_A(m)(n)(p)), \refl_A(m))
      \end{multline*}
    \item
      \textbf{Inverse-left.}
      ``$p^{-1} \concat p = \refl$''.
      \begin{multline*}
        \textstyle
        \prod_{m{:}A}\prod_{n{:}A}\prod_{p{:}\Id_A(m,n)}\\
        \Id_{\Id_A(n,n)}(\trans_A(n)(m)(n)(\sym_A(m)(n)(p))(p), \refl_A(n))
      \end{multline*}
    \item
      \textbf{Unit-right.}
      ``$p \concat \refl = p$''.
      \begin{multline*}
        \textstyle
        \prod_{m{:}A}\prod_{n{:}A}\prod_{p{:}\Id_A(m,n)}\Id_{\Id_A(m,n)}(\trans_A(m)(n)(n)(p)(\refl_A(n)), p)
      \end{multline*}

    \item
      \textbf{Unit-left.}
      ``$\refl \concat p = p$''.
      \[
        \textstyle
        \prod_{m{:}A}\prod_{n{:}A}\prod_{p{:}\Id_A(m,n)}\Id_{\Id_A(m,n)}(\trans_A(m)(m)(n)(\refl_A(m))(p), p)
      \]

    \item
      \textbf{Associativity.}
      ``$(p \concat q) \concat r = p \concat (q \concat r)$''.
      \begin{multline*}
        \textstyle
        \prod_{l{:}A}\prod_{m{:}A}\prod_{n{:}A}\prod_{o{:}A}\prod_{p{:}\Id_A(l,m)}\prod_{q{:}\Id_A(m,n)}\prod_{r{:}\Id_A(n,o)}
        \\
        \Id_{\Id_A(l,o)}(\trans_A(l)(n)(o)(\trans_A(l)(m)(n)(p)(q))(r),
        \\
        \trans_A(l)(m)(o)(p)(\trans_A(m)(n)(o)(q)(r)))
      \end{multline*}
  \end{itemize}
\end{task}

\section{Elementary Math}

\subsection{Commutativity of Addition}

Assume that our implementation of $+$ has the following computational behaviors:
\begin{itemize}
  \item $\zero + b \jdeq b$.
  \item $\suc(a) + b \jdeq \suc(a + b)$.
\end{itemize}

\begin{task}
  Write code of type $\prod_a a + \zero =_\nat a$ (and probably name it).
\end{task}
\begin{task}
  Write code of type $\prod_a \prod_b a + \suc(b) =_\nat \suc(a + b)$ (and name it).
\end{task}
\begin{task}
  Finally, write code of type $\prod_a \prod_b a + b =_\nat b + a$
  which uses the previous two tasks as subroutines.
\end{task}

\subsection{Injectivity of Successor}

\begin{task}
  Prove (a.k.a. write code of type)
  \[
    \prod_a \prod_b (\suc(a) =_\nat \suc(b)) \to (a =_\nat b).
  \]
  \begin{hint}
    Write a function $f$ such that $f(\suc(a)) \jdeq a$.
  \end{hint}
\end{task}

\subsection{Non-degeneracy of Natural Numbers}

\begin{task}
  Prove $\prod_a (\suc(a) =_\nat \zero) \to \bot$.
  \begin{hint}
    Can you come up with a function $f$ and a path $p : a =_\nat b$ such that
    $f(a) \jdeq \top$ but $f(b) \jdeq \bot$?
    What can you do with such $f$ and $p$?
  \end{hint}
\end{task}

\end{document}
